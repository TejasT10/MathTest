For point \(B\) to be symmetric to point \(A\) about point \(C\), \(C\) has to be the midpoint of the segment joining points \(A\) and 
\(B\). We know the midpoint formula which tells us that the coordinates of the midpoint are the averages of the coordinates of both points:
\[M(x,y) = \frac{A(x_1,y_1)}{2} + \frac{B(x_2,y_2)}{2} = \left(\frac{x_1+x_2}{2},\frac{y_1+y_2}{2}\right)\]

Putting \(M = C = (1,-1)\) and \(A = (-3,1)\) in the above equation and letting the coordinates of \(B\) be \((x,y)\), we get:
\[(1,-1) = \left(\frac{-3+x}{2},\frac{1+y}{2}\right)\]

This gives us two separate equations to solve, one each for \(x\) and \(y\) respectively: \(1 = \frac{-3+x}{2}\) and \(-1 = \frac{1+y}{2}\). Multiplying by 2 and solving both equations independently, we have:
\\
\(1 = \frac{-3+x}{2} \Rightarrow 2\times1 = -3 + x \\ \Rightarrow 2 + 3 = (-3) + x + 3 = x \\ \Rightarrow x = 5\)
\\
\(-1 = \frac{1+y}{2} \Rightarrow 2\time(-1) = 1+y \\ \Rightarrow -2 - 1 = 1 + y - 1 = y \\ \Rightarrow y = -3\)
\\
Thus, we get our coordinates as \(B = (5,-3)\), which gives the final answer \(= 5 + (-3) = 2\).
