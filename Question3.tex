Suppose \(A\) is a square matrix such that  \(AA^T = I\) and \(\det(A) < 0\). Show that \(\det(A+I) = 0\). 
\subsection*{Answer 3}
We are given matrices (of order \(n\)). We wish to prove that \(\det(A+I)=0\). Note that we haven't been asked to find the value of \(\det(A+I)\); we've been given its value directly so we should try and use that information. 
\\
We know that matrix product is distributive over the determinant (mod) operation, \textit{i.e.} \(\det(P\times Q)=\det(P)\times\det(Q) \) where \(P, Q\) are both square matrices (of same order).
\\
So if we can choose our \(P\) and \(Q\) suitably, we can accomplish the task. We also have the fact that transposing doesn't affect the determinant, \(\Rightarrow \det(A) = \det(A^T)\). 
\\
Using the above two facts on the equation \(AA^T = I\), we get \[\det(AA^T) = \det(A)\times \det(A^T) = \det(A)\times \det(A) = \det(A)^2\] we know \(\det(I)=1\). Thus, equating the two expression we obtain \(\det(A)^2 = 1 \Rightarrow \det(A)=-1\) as \(\
\det(A) < 0\) was given to us.
\\
In original multiplicative identity, if we take \(P=A+I\) and \(Q=A^T\) then we get:
\[\det((A+I)(A^T)) = \det(AA^T + IA^T) = \det(A+I)\times \det(A) \]
%\(\det(A+I)\)
Note that \(AA^T=I\) so we have \(AA^T + IA^T = I + A^T\). Also, since \((I+A^T)^T = I^T + (A^T)^T = I + A\) we get \(\det(I+A^T) = \det(I+A)\).
\\
Thus, substituting this back into the equation we get \(\det(I+A^T) = \det(I+A) \times \det(A)\). Let if possible \(\det(I+A)\) not be zero. Then as \(\det(I+A^T) = \det(I+A)\) we can simply divide both sides of our equation by \(\det(I+A)\) to get \(\det(A) = 1\) which is a contradiction as \(\det(A)<0\).
\\
Thus, \(\det(I+A)\) has to be equal to zero. Hence proved.
